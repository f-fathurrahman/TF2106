\documentclass[10pt,english,twocolumn,fleqn]{extarticle}

%\usepackage[T1]{fontenc}
%\usepackage[latin9]{inputenc}

\usepackage{geometry}
\geometry{verbose,tmargin=1.0cm,bmargin=1.5cm,lmargin=0.8cm,rmargin=0.8cm}
\setlength{\parskip}{\smallskipamount}
\setlength{\parindent}{0pt}
\usepackage{amsmath}
\usepackage{amstext}
\usepackage{amsbsy}
\usepackage{mathrsfs}
\usepackage{graphicx}

%\usepackage{unicode-math}
%\setmainfont{Linux Libertine}
%\setmathfont{XITS Math}

\usepackage[libertine]{newtxmath}
\usepackage[no-math]{fontspec}
\setmainfont{Linux Libertine O}

\usepackage{babel}

\def\rc{{\mbox{$\resizebox{.09in}{.08in}{\includegraphics[trim= 1em 0 14em 0,clip]{../common/ScriptR}}$}}}
\def\brc{{\mbox{$\resizebox{.09in}{.08in}{\includegraphics[trim= 1em 0 14em 0,clip]{../common/BoldR}}$}}}

\newcommand{\dd}{\textnormal{d}}


\begin{document}

\title{Catatan Kuliah TF2106: Magnetostatika}
\author{Fadjar Fathurrahman}
\date{2016}
\maketitle

\section{Gaya magnetik}
Hukum gaya Lorentz:
\begin{equation}
\mathbf{F}_{\textnormal{mag}} = Q(\mathbf{v}\times\mathbf{B})
\end{equation}

Gaya magnetik tidak melakukan kerja:
\begin{eqnarray*}
\dd W_{\textnormal{mag}} & = & \mathbf{F}_{\textnormal{mag}} \cdot \dd \mathbf{l} \\
 & = & Q (\mathbf{v} \times \mathbf{B}) \cdot \mathbf{v}\, \dd t \\
 & = & 0
\end{eqnarray*}

\section{Arus}
Arus listrik:
\begin{equation*}
I = \frac{\delta Q}{\delta t}
\end{equation*}

Gaya magnet pada arus garis, arus permukaan, dan arus volume
\begin{eqnarray}
\mathbf{F}_{\textnormal{mag}} = \int (\mathbf{v} \times \mathbf{B})\,\lambda\,\dd l &\,& \textnormal{(garis)} \\
\mathbf{F}_{\textnormal{mag}} = \int (\mathbf{v} \times \mathbf{B})\,\sigma\,\dd a &\,& \textnormal{(permukaan)} \\
\mathbf{F}_{\textnormal{mag}} = \int (\mathbf{v}\times\mathbf{B})\,\rho\,\dd \tau &\,& \textnormal{(volume)}
\end{eqnarray}

Persamaan kontinuitas:
\begin{equation}
\nabla \cdot \mathbf{J} = - \frac{\partial \rho}{\partial t}
\end{equation}

\section{Hukum Biot-Savart}
Medan magnet dari arus tunak pada garis adalah:
\begin{equation}
\mathbf{B}(\mathbf{r}) = \int \frac{\mu_0}{4\pi} \frac{\mathbf{I}\times\hat{\brc}}{\rc^2}\,\dd l'
\end{equation}
Konstanta $\mu_{0} = 4\pi\times10^{-7}\,\mathrm{N/A^2}$ adalah permeabilitas vakum.
Satuan dari $\mathbf{B}$ adalah newton per ampere-meter atau tesla.

Untuk arus permukaan hukum Biot-Savart dinyatakan sebagai:
\begin{equation}
\mathbf{B}(\mathbf{r}) = \int \frac{\mu_0}{4\pi} \frac{\mathbf{K}(\mathbf{r}')\times\hat{\brc}}{\rc^2}\,\dd l'
\end{equation}
Sedangkan untuk arus volume hukum Biot-Savart dinyatakan sebagai:
\begin{equation}
\mathbf{B}(\mathbf{r}) = \int \frac{\mu_0}{4\pi} \frac{\mathbf{J}(\mathbf{r}')\times\hat{\brc}}{\rc^2}\,\dd l'
\end{equation}

\section{Divergensi dan curl dari medan magnet}
Divergensi dari medan magnet adalah nol.
\begin{equation}
\nabla \cdot \mathbf{B} = 0
\end{equation}

Curl dari $\mathbf{B}$ adalah
\begin{equation}
\nabla \times \mathbf{B} = \mu_0 \mathbf{J}
\end{equation}
Persamaan ini juga dikenal dengan nama Hukum Ampere.
Dalam bentuk integral Hukum Ampere dapat dituliskan sebagai:
\begin{equation}
\oint \mathbf{B}\cdot\dd\mathbf{l} = \mu_0 I_{\mathrm{enc}}
\end{equation}


\section{Potensial vektor magnetik}
Potensial vektor magnetik $\mathbf{A}$ diperoleh dari persamaan
\begin{equation}
\mathbf{B} = \nabla \times \mathbf{A}
\end{equation}
Persamaan ini memberikan curl dari $\mathbf{A}$ akan tetapi tidak memberikan
informasi mengenai divergensi dari $\mathbf{A}$. Kita bebas memilih nilai divergensi
$\mathbf{A}$ dan nol merupakan pilihan yang paling sederhana.
\begin{equation}
\nabla \cdot \mathbf{A} = 0
\end{equation}
Dengan pemilihan ini, Hukum Ampere dapat dituliskan menjadi
\begin{equation}
\nabla^2 \mathbf{A} = -\mu_0 \mathbf{J}
\end{equation}
Dengan asumsi bahwa $\mathbf{J}$ mendekati nol pada tak hingga, dapat diperoleh
solusi
\begin{equation}
\mathbf{A}(\mathbf{r}) = \frac{\mu_0}{4\pi} \int \frac{\mathbf{J}(\mathbf{r}')}{\rc}\,\dd \tau'
\end{equation}

Untuk arus garis:
\begin{equation}
\mathbf{A}(\mathbf{r}) = \frac{\mu_0}{4\pi} \int \frac{\mathbf{I}}{\rc}\, \dd l' =
\frac{\mu_0 I}{4\pi} \int \frac{1}{\rc}\,\dd \mathbf{l}'
\end{equation}
sedangkan untuk arus permukaan:
\begin{equation}
\mathbf{A}(\mathbf{r}) = \frac{\mu_0}{4\pi} \int \frac{\mathbf{K}}{\rc}\, \dd a'.
\end{equation}

\section{Syarat-syarat batas}
Komponen medan magnetik yang tegak lurus terhadap permukaan:
\begin{equation}
B^{\bot}_{\mathrm{above}} = B^{\bot}_{\mathrm{below}}
\end{equation}

Komponen medan magnetik yang sejajar dengan permukaan:
\begin{equation}
\mathbf{B}^{\parallel}_{\mathrm{above}} - \mathbf{B}^{\parallel}_{\mathrm{below}} = \mu_0 K
\end{equation}

Dua kondisi tersebut dapat dituliskan sebagai:
\begin{equation}
\mathbf{B}_{\mathrm{above}} - \mathbf{B}_{\mathrm{below}} = \mu_0 \left( \mathbf{K} \times \hat{n} \right)
\end{equation}

Potensial vektor bersifat kontinu di semua perbatasan.
\begin{equation}
\mathbf{A}_{\mathrm{above}} = \mathbf{A}_{\mathrm{below}}
\end{equation}


\section{Ekspansi multipol dan potensial vektor}

Dengan menggunakan ekspansi multipol:
\begin{equation}
\frac{1}{\rc} = \frac{1}{r} \sum_{n=0}^{\infty} \left(\frac{r'}{r}\right)^{n}
P_{n}(\cos\alpha)
\end{equation}
dengan $\alpha$ merupakan sudut antara $\mathbf{r}$ dan $\mathbf{r}'$

Potensial vektor dapat dituliskan menjadi:
\begin{align*}
\mathbf{A}(\mathbf{r}) & = \frac{\mu_0 I}{4\pi} \oint \frac{1}{\rc}\dd\mathbf{l}' \\
& = \frac{\mu_0 I}{4\pi} \frac{1}{r^{n+1}} \oint (r')^{n} P_{n}(\cos\alpha)\,\dd\mathbf{l}'
\end{align*}
Tinjau suku-suku $1/r^{n}$:
\begin{align*}
\mathbf{A}_{\mathrm{mon}}(\mathbf{r}) & = \frac{\mu_0 I}{4\pi}\frac{1}{r}\oint\dd\mathbf{l}' \\
\mathbf{A}_{\mathrm{dip}}(\mathbf{r}) & = \frac{\mu_0 I}{4\pi}\frac{1}{r^2}
\oint r'\,\cos\,\alpha\,\dd\mathbf{l}' \\
\mathbf{A}_{\mathrm{quad}}(\mathbf{r}) & = \frac{\mu_0 I}{4\pi}
\end{align*}

Suku monopol magnetik selalu bernilai $\mathbf{0}$, karena
\begin{equation}
\oint \dd\mathbf{l}' = \mathbf{0}
\end{equation}

Suku dominan adalah dipol
\begin{equation}
\mathbf{A}_{\mathbf{dip}}(\mathbf{r}) = \frac{\mu_0 I }{4\pi r^2} \oint r' \cos\,\alpha\,\dd\mathbf{l}'
= \frac{\mu_0 I}{4\pi r^2} \oint ( \hat{\mathbf{r}} \cdot \mathbf{r}' )\,\dd\mathbf{l}'
\end{equation}
dengan menggunakan
\begin{equation}
\oint ( \hat{\mathbf{r}} \cdot \mathbf{r}' )\,\dd\mathbf{l}' = -\hat{\mathbf{r}}\times\int\dd\mathbf{a}'
\end{equation}
Maka persamaan untuk dipol dapat dituliskan menjadi:
\begin{equation}
\mathbf{A}_{\mathrm{dip}} = \frac{\mu_0}{4\pi}\frac{\mathbf{m} \times \hat{\mathbf{r}}}{r^2}
\end{equation}
di mana $\mathbf{m}$ adalah momen dipol magnetik:
\begin{equation}
\mathbf{m} \equiv I\int\dd\mathbf{a} = I\mathbf{a}
\end{equation}



\end{document}
