\documentclass[10pt,english,twocolumn]{extarticle}

%\usepackage[T1]{fontenc}
%\usepackage[latin9]{inputenc}

\usepackage{geometry}
\geometry{verbose,tmargin=1.0cm,bmargin=1.5cm,lmargin=0.8cm,rmargin=0.8cm}
\setlength{\parskip}{\smallskipamount}
\setlength{\parindent}{0pt}
\usepackage{amstext}
%\usepackage{amsfonts}
\usepackage{amsbsy}
\usepackage{mathrsfs}

\usepackage{unicode-math}
\setmainfont{Linux Libertine}
\setmathfont{XITS Math}
%\setmathfont[version=setB,StylisticSet=1]{XITS Math}

\usepackage{babel}

\newcommand{\dd}{\textnormal{d}}

\begin{document}

\title{Magnetostatik}
\author{Fadjar Fathurrahman}
\date{2016}
\maketitle

\section{Gaya magnetik}

Gaya magnet yang dirasakan oleh suatu muatan $Q$ yang bergerak dengan kecepatan $\mathbf{v}$
pada suatu medan magnet $\mathbf{B}$ adalah
\begin{equation}\label{eq:LorentzEq}
\mathbf{F}_{\textnormal{mag}} = Q(\mathbf{v}\times\mathbf{B})
\end{equation}
Persamaan \ref{eq:LorentzEq} dikenal dengan nama hukum gaya Lorentz.
Jika muatan juga dikenai medan listrik $\mathbf{E}$ selain medan magnet $\mathbf{B}$,
maka total gaya pada muatan $Q$ adalah:
\begin{equation}
\mathbf{F} = Q\left[ \mathbf{E} + (\mathbf{v}\times\mathbf{B}) \right]
\end{equation}

Jika $\mathbf{Q}$ bergerak sejauh $\dd \mathbf{l} = \mathbf{v}\, \dd t$ maka kerja yang
dilakukan adalah
\begin{eqnarray*}
\dd W_{\textnormal{mag}} & = & \mathbf{F}_{\textnormal{mag}} \cdot \dd \mathbf{l} \\
 & = & Q (\mathbf{v} \times \mathbf{B}) \cdot \mathbf{v}\, \dd t \\
 & = & 0
\end{eqnarray*}
\textit{Gaya magnetik tidak melakukan kerja}.

\section{Arus}

Arus didefinisikan sebagai jumlah muatan yang mengalir persatuan waktu pada
suatu titik. Arus diukur dalam satuan coulomb-per-detik atau ampere:
\begin{equation}
1\,\textnormal{A} = 1\,\textnormal{C/s}
\end{equation}

Gaya magnetik pada suatu segmen kawat berarus adalah:
\begin{eqnarray*}
\mathbf{F}_{\textnormal{mag}} & = & \int (\mathbf{v} \times \mathbf{B})\,\dd q \\
& = & \int (\mathbf{v} \times \mathbf{B})\, \lambda\, \dd l \\
& = & \int (\mathbf{I} \times \mathbf{B})\,\dd l
\end{eqnarray*}
Karena arus $\mathbf{I}$ dan segment kawat $\dd\mathbf{l}$ memiliki arah yang sama,
\begin{equation}
\mathbf{F}_{\textnormal{mag}} = \int I (\dd \mathbf{l} \times \mathbf{B})
\end{equation}

Arus yang mengalir pada suatu permukaan dapat dideskripsikan dengan rapat arus
permukaan, $\mathbf{K}$.
Jika rapat muatan permukaan adalah $\sigma$ dan kecepatannya adalah $\mathbf{v}$, maka
rapat arus permukaan dapat dinyatakan dengan
\begin{equation}
\mathbf{K} = \sigma \mathbf{v}
\end{equation}
Maka, gaya magnet pada arus permukaan adalah
\begin{equation}
\mathbf{F}_{\textnormal{mag}} = \int (\mathbf{v} \times \mathbf{B})\,\sigma\,\dd S
\end{equation}

Untuk arus yang mengalir pada suatu volume dapat dideskripsikan dengan rapat muatan
volume, $\mathbf{J}$. Jika rapat muatan volume adalah $\rho$ dan kecepatannya adalah
$\mathbf{v}$, maka
\begin{equation}
\mathbf{J} = \rho \mathbf{v}
\end{equation}
Gaya magnetik pada arus volume adalah
\begin{eqnarray}
\mathbf{F}_{\textnormal{mag}} & = & \int (\mathbf{v}\times\mathbf{B})\,\rho\,\dd V \\
& = & \int (\mathbf{J}\times\mathbf{B})
\end{eqnarray}

\textbf{Persamaan kontinuitas}

Arus total yang mengalir melalui suatu permukaan $\mathcal{S}$ adalah
\begin{equation}
I = \int_{\mathcal{S}} J\, \dd A_{\bot} = \int_{\mathcal{S}} \mathbf{J} \cdot \dd \mathbf{S}
\end{equation}
Persamaan kontinuitas:
\begin{equation}
\nabla \cdot \mathbf{J} = - \frac{\partial \rho}{\partial t}
\end{equation}

\section{Hukum Biot-Savart}
Medan magnet dari arus tunak pada garis adalah:
\begin{equation}
\mathbf{B}(\mathbf{r}) = \frac{\mu_0}{4\pi} \frac{\mathbf{I}\times\hat{\mathscr{r}}}{\mathscr{r}^2}
\end{equation}

\begin{equation}
\mathcal{abcdefghijklmnopqrstuvwxyz}
\end{equation}

\begin{equation}
\mathscr{abcdefghijklmnopqrstuvwxyz}
\end{equation}

\end{document}
