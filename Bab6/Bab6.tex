\documentclass[10pt,english,twocolumn,fleqn]{extarticle}

%\usepackage[T1]{fontenc}
%\usepackage[latin9]{inputenc}

\usepackage{geometry}
\geometry{verbose,tmargin=1.0cm,bmargin=1.5cm,lmargin=0.8cm,rmargin=0.8cm}
\setlength{\parskip}{\smallskipamount}
\setlength{\parindent}{0pt}
\usepackage{amsmath}
\usepackage{amstext}
\usepackage{amsbsy}
\usepackage{mathrsfs}
\usepackage{graphicx}

%\usepackage{unicode-math}
%\setmainfont{Linux Libertine}
%\setmathfont{XITS Math}

\usepackage[libertine]{newtxmath}
\usepackage[no-math]{fontspec}
\setmainfont{Linux Libertine O}

\usepackage{babel}

\def\rc{{\mbox{$\resizebox{.09in}{.08in}{\includegraphics[trim= 1em 0 14em 0,clip]{../common/ScriptR}}$}}}
\def\brc{{\mbox{$\resizebox{.09in}{.08in}{\includegraphics[trim= 1em 0 14em 0,clip]{../common/BoldR}}$}}}

\newcommand{\dd}{\textnormal{d}}


\begin{document}

\title{Catatan Kuliah TF2106: Medan Magnet dalam Material}
\author{Fadjar Fathurrahman}
\date{2016}
\maketitle

\section{Magnetisasi: diamagnet, paramagnet, dan feromagnet}
Jika kita mengamati material magnetik pada skala atomik, kita akan menemukan arus
listrik yang disebabkan oleh elektron yang mengorbit inti atom dan berputar
pada sumbunya. Untuk keperluan makroskopik, loop arus tersebut dapat dianggap
sebagai dipol magnetik. Pada keadaan normal, dipol ini saling menghilangkan satu
sama lainnya karena orientasi acak dari atom. Akan tetapi, ketika suatu medan
magnetik diberikan, akan terdapat penjajaran dipol-dipol, dan material tersebut
menjadi termagnetisasi atau terpolarisasi secara magnetik.

Berbeda dengan fenomena polarisasi listrik, di mana dipol listrik hampir selalu mengarah
ke arah yang sama dengan medan listrik $\mathbf{E}$, pada magnetisasi beberapa
material akan menghasilkan magnetisasi searah dengan $\mathbf{B}$.
Tipe material ini dinamakan paramagnet.
Beberapa material akan menghasilkan magnetisasi berlawanan arah dan tipe material
ini dinamakan diamagnet.
Ketika medan magnet dihilangkan, maka magnetisasi dari paramagnet dan diamagnet
akan hilang. Jenis material yang disebut feromagnet, tetap memiliki magnetisasi
meskipun medan magnet telah dihilangkan.

\section{Torsi dan gaya pada dipol magnetik}

Torsi $\mathbf{N}$ pada dipol magnetik dapat dituliskan sebagai:
\begin{equation}
\mathbf{N} = \mathbf{m} \times \mathbf{B}
\end{equation}

Untuk loop arus yang sangat kecil, gaya yang dirasakan oleh dipol
adalah
\begin{equation}
\mathbf{F} = \nabla(\mathbf{m}\cdot\mathbf{B})
\end{equation}

\section{Magnetisasi}
Magnetisasi didefinisikan sebagai momen dipol magnetik per satuan volume.
Magnetisasi disimbolkan dengan $\mathbf{M}$ dan memainkan peran analog
dengan polarisasi $\mathbf{P}$ pada elektrostatika.

\section{Medan dari suatu material termagnetisasi}
Dalam suatu material yang termagnetisasi, tiap elemen volume $\dd\tau'$
memiliki moment dipol $\mathbf{M}\dd\tau'$, sehingga total potensial vektor
dari material ini adalah
\begin{equation}
\mathbf{A}(\mathbf{r}) = \frac{\mu_0}{4\pi} \int
\frac{\mathbf{M}(\mathbf{r}')\times\hat{\brc}}{\rc^2}\dd\tau'
\end{equation}
Setelah menggunakan beberapa manipulasi matematis, persamaan ini dapat dituliskan
menjadi
\begin{equation}\label{eq:magnetized}
\mathbf{A}(\mathbf{r}) = \frac{\mu_0}{4\pi}\int_{\mathcal{V}}\frac{\mathbf{J}_{b}(\mathbf{r}')}{\rc}\dd\tau'
+ \frac{\mu_0}{4\pi}\oint_{\mathcal{S}} \frac{\mathbf{K}_b(\mathbf{r}')}{\rc}\dd a'
\end{equation}
dengan $\mathbf{J}_b$ dan $\mathbf{K}_b$ menyatakan densitas arus terikat (\textit{bound current})
pada volume dan permukaan dari material yang bersangkutan.
\begin{align}
\mathbf{J}_b & = \nabla \times \mathbf{M} \\
\mathbf{K}_b & = \mathbf{M} \times \hat{\mathbf{n}}
\end{align}

Arti dari persamaan \ref{eq:magnetized}: efek dari magnetisasi adalah menghasilkan
arus terikat dengan kerapatan $\mathbf{J}_b = \nabla\times\mathbf{M}$ di dalam material dan
$\mathbf{K}_b = \mathbf{M}\times\hat{\mathbf{n}}$ pada permukaan material.

\section{Medan bantu (\textit{auxiliary field}) H}
Total kerapatan arus dapat dituliskan sebagai:
\begin{equation}
\mathbf{J} = \mathbf{J}_b + \mathbf{J}_f
\end{equation}
Hukum Ampere dapat dituliskan menjadi
\begin{equation}
\nabla \times \mathbf{B} = \mu_0 \mathbf{J} = \mu_0 (\mathbf{J}_b + \mathbf{J}_f)
= \mu_0 \left( (\nabla \times \mathbf{M}) + \mathbf{J}_f \right)
\end{equation}
atau:
\begin{equation}
\nabla\times\left( \frac{\mathbf{B}}{\mu_0} - \mathbf{M} \right) = \mathbf{J}_f
\end{equation}
Dengan menggunakan definisi
\begin{equation}
\mathbf{H} \equiv \frac{\mathbf{B}}{\mu_0} - \mathbf{M},
\end{equation}
Hukum Ampere dapat dituliskan sebagai
\begin{equation}
\nabla \times \mathbf{H} = \mathbf{J}_f
\end{equation}
atau dalam bentuk integral
\begin{equation}
\oint \mathbf{H}\cdot\dd\mathbf{l} = I_{f,\mathrm{enc}}
\end{equation}
di mana $I_{f,\mathrm{enc}}$ adalah arus bebas yang melalui loop Amperian.

\section{Syarat-syarat batas}
\begin{equation}
H^{\bot}_{\mathrm{above}} - H^{\bot}_{\mathrm{below}} = -\left( M^{\bot}_{above}
- M^{\bot}_{below} \right)
\end{equation}

\begin{equation}
\mathbf{H}^{\bot}_{\mathrm{parallel}} - \mathbf{H}^{\parallel}_{\mathrm{below}} =
\mathbf{K}_f \times \hat{\mathbf{n}}
\end{equation}



\end{document}
